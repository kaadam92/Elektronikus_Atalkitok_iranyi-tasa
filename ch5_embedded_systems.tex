\section{Beágyazott irányító rendszerek}
\vspace{5cm}
\subsection{Beágyazott irányító rendszerek típusai}

\subsection{Mikrovezérlő alapú rendszerek}

\subsubsection{Gyakran használt mikrovezérlő perifériák}

\subsection{Beágyazott kommunikációs interfészek}

\subsection{Digitális kommunikációs vonalak felépítése}

Digitális eszközök közötti adatcserére az iparban szabványoks kommunikációs protokollokat, kommuinkációs formákat haszánlunk. Ezek rendkívül sokrétűek, az olvasó is találkozhat velük a mindennapjaiban, legyen szó egy FM rádióadásról, a GSM hálózatról, vagy egy Ethernet kapcsolatról. Ezek a kommunkációs eljárások különböző, standard rétegeket implementálnak, melyek akár fel is cserélhetőek egymás között. (Pl. TCP/IP alapú kapcsolat megvalósítható WiFi-n és vezetlekes Ethernet hálózaton is) Ilyen modellszerű leírást ad meg az Open Systems Interconnection model (OSI model) is.

\paragraph{Layer 1: Fizikai réteg}

A fizikai réteg azt határozza meg, hogy milyen fizikai közegen keresztül áramlik az információ az eszközök között. A fizikai réteg specifikálja, hogy vezetékes, vagy vezeték nélküli kapcsolatról van-e szó, az átvitt jelek sázmát, az elvárt feszültésgszinteket és időzítéseket, a moduláció módját. A fizikai réteg dönti el, hogy az adat mely irányba áramolhat, simplex (egy irányú), half duplex (időosztáyban két irányú), vagy full duplex (folyamatosan két irányú) kommunikációról beszélünk-e.

\paragraph{Layer 2: Adatkapcsolati réteg}

\begin{example}[\quad \large Zavarérzékenység]

A tanult soros kommunikációk fizikai rétegei közül melyek hasonlóak zavarérzékenység, ill. logikai működés szempontjából?


\end{example}

\subsubsection{I2C}

Az $I^2C$ (Inter-Integrated Circuit) egy szinkon, csomagkapcsolt soros kommunikációs forma, melyet 1982-ben definiált a Philips Semiconductor (ma NXP). Igen elterjed, általában nyomtatott áramkörön belül, a különböző digitális eszközök között használt kommunikációs forma. Több, kompatibilis változata is létezik, iylen pl. az SMBus.

Leggyakrabban külsö ADC-k, EEPROM-ok és egyéb Flash memóriák, digitális érzékelők,  vagy egyszerű kijelzők vezérlésésre alkalmazzák.

Az I2C kommunikáció fizikai rétege 2 vezetékeből áll, melyre a perifériák open-collectoros kimenetekkel csatlakoznak, emiatt a busz vezetékeit fel kell húzni tápfeszültésgre. A tápfeszültségre felhúzú ellenállás méretét a kommunikációs sebesség, illetve a buszon található eszközök kapcitív terhelése határozza meg

\begin{example}[\quad \large I2C 1]

Ismertesse az I2C busz adatkapcsolati rétegét! Rajzoljon fel egy elrendezést 2 potenciális mester és 2 szolga résztvevővel! Vázolja fel a busz jeleit, ha a mester a 3 című eszközre egy 0 értékű és 255 értékű byte-ot ír és az eszköz képes a vételre!


\tcbline
\vspace{1mm}

\solution

\end{example}
\begin{example}[\quad \large I2C 2]

Ismertesse az I2C busz adatkapcsolati rétegét! Rajzoljon fel egy elrendezést 1 mester és 2 szolga résztvevővel! Vázolja fel a busz jeleit, ha a mester a 3 című eszközre egy 0 értékű byte-ot ír és az eszköz képes a vételre!


\tcbline
\vspace{1mm}

\solution

\end{example}
\begin{example}[\quad \large I2C 3]

Ismertesse az I2C busz adatkapcsolati rétegét! Rajzoljon fel egy elrendezést 1 mester és 2 szolga résztvevővel! Vázolja fel a busz jeleit, ha a mester a 127 című eszközre egy 0 értékű byte-ot ír és az eszköz képes a vételre!

\tcbline
\vspace{1mm}

\solution

\end{example}
\begin{example}[\quad \large I2C 4]

Ismertesse egy I2C buszos EEPROM egyetlen adatbájtja olvasásának fázisait.

\tcbline
\vspace{1mm}

\solution

\end{example}
\begin{example}[\quad \large I2C 5]

Rajzoljon fel egy elrendezést 1 mester és 2 szolga résztvevővel! Mi történik, ha a mester tévedésből két azonosra állított című eszközből olvas, ha az egyik eszközből olvasandó adat 0xaa, a másikból 0x55?

\tcbline
\vspace{1mm}

\solution

\end{example}


\subsubsection{SPI}

\begin{example}[\quad \large SPI 1]

Rajzoljon fel egy mestert és két szolgát tartalmazó rendszert! 

\tcbline
\vspace{1mm}

\solution

\end{example}
\begin{example}[\quad \large SPI 2]

Egy abszolút pozíció érzékelőt SSI interfészen illesztünk a folyamatirányító számítógéphez. Ismertesse a bekötéshez szükséges vezetékek szerepét! Melyik pillanatban érvényes pozíciót kapjak meg az irányító egység?

\tcbline
\vspace{1mm}

\solution

\end{example}
\begin{example}[\quad \large SPI 3]

Adja meg egy 4 bites adatok duplex átadásakor az SCLK, MOSI, MISO, CS (SL) jelek időfüggvényét!

\tcbline
\vspace{1mm}

\solution

\end{example}

\subsubsection{UART}

\begin{example}[\quad \large UART 1]

Rajzoljon fel egy 4 állomásos RS485-ös rendszert! 

\tcbline
\vspace{1mm}

\solution

\end{example}
\begin{example}[\quad \large UART 2]

Hasonlítsa össze az RS422 és az RS485 szabványokat!

\tcbline
\vspace{1mm}

\solution

\end{example}
\begin{example}[\quad \large UART 3]

Aszinkron soros kommunikáció alapszintű protokollja. Pl.: 7E2. 

\tcbline
\vspace{1mm}

\solution

\end{example}
\begin{example}[\quad \large UART 4]

Két processzor között aszinkron kommunikációt valósítunk meg. Megfelelnek-e a ±2,5\%-os pontosságú órajel-generátorok, ha a kommunikációs mód 8E1? Mikor lehet szükség egynél több stop bitre?

\tcbline
\vspace{1mm}

\solution

\end{example}
\begin{example}[\quad \large UART 5]

Rajzolja fel egy két részvevős RS485 rendszer vezetéke GND-hez képesti feszültségének időfüggvényét, amikor az egyik résztvevő 0x5A kódot küld 7O2 kommunikációs módban!

\tcbline
\vspace{1mm}

\solution

\end{example}
\begin{example}[\quad \large UART 6]

Rajzoljon fel egy 4 résztvevős RS485 szabvány szerinti kommunikációs rendszert! Mit érzékelnek a vevők, ha egyik részvevő sem ad?

\tcbline
\vspace{1mm}

\solution

\end{example}

\subsubsection{CAN}

\begin{example}[\quad \large CAN 1]

Mi a soros és a párhuzamos visszaverődés-mentesítés? Mikor melyiket célszerű alkalmazni?

\tcbline
\vspace{1mm}

\solution

\end{example}
\begin{example}[\quad \large CAN 2]

Egy CAN buszon két eszköz egyszerre kezd adni egy-egy standard ID-jű üzenetet. Az első eszköz üzenetének azonosítója 120, a másiké 64. Rajzolja fel az első eszköz TX és RX jelének első 12 bitjét!

\tcbline
\vspace{1mm}

\solution

\end{example}
\begin{example}[\quad \large CAN 3]

Egy CAN buszon két eszköz egyszerre kezd adni egy-egy standard ID-jű üzenetet. Az első eszköz üzenetének azonosítója 2047, a másiké 0. Rajzolja fel az első eszköz TX és RX jelének első 8 bitjét! Rajzoljon fel egy 4 résztvevős RS485 szabvány szerinti kommunikációs rendszert!

\tcbline
\vspace{1mm}

\solution

\end{example}
\begin{example}[\quad \large CAN 4]

CAN buszon az 1. állomás átviteli sebességét 500kBaudra állítottuk, a 2. állomás sebességét 1MBaud-ra. Mi lesz a 2. állomás TxD és RxD jele az első 10us-ban, ha az 1. állomás ID=3 azonosítójú üzenetet kezd el küldeni?

\tcbline
\vspace{1mm}

\solution

\end{example}
\begin{example}[\quad \large CAN 5]

Mit jelent a bit szintű arbitráció? A CAN üzenet melyik része az arbitrációs mező?

\tcbline
\vspace{1mm}

\solution

\end{example}
\begin{example}[\quad \large CAN 6]

Hány bites a DLC mező. Miért?

\tcbline
\vspace{1mm}

\solution

\end{example}
\begin{example}[\quad \large CAN 7]

Hány bites a standard ID? Mit jelent az, hogy a CAN üzenet-orientált?

\tcbline
\vspace{1mm}

\solution

\end{example}
\begin{example}[\quad \large CAN 8]

Ismertesse a CAN (A) üzenet felépítését! Hogyan változik meg a buszon mérhető jelsorozat, ha az egyik aktív ill. passzív vevő a CRC-t hibásnak érzékeli? 

\tcbline
\vspace{1mm}

\solution

\end{example}
\begin{example}[\quad \large CAN 9]

Mi az “error frame” és mikor adja egy résztvevő?

\tcbline
\vspace{1mm}

\solution

\end{example}
\begin{example}[\quad \large CAN 10]

Két processzor között CAN kommunikációt valósítunk meg. Megfelelnek-e a ±2,5\%-os pontosságú órajel-generátorok? 

\tcbline
\vspace{1mm}

\solution

\end{example}


\vspace{-1.5mm}
\newpage