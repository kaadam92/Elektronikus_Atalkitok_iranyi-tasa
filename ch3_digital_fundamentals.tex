\section{Digitális technika alpajai}
\vspace{5cm}

Az előző fejezetben megismert analóg elektronikai ismeretek segítségével sok irányítási és szabályozási probléma megoldható, azonban ezek használata ma már csak szék és speciális körökre korlátozódik, az analóg szabályzások helyét a digitális megoldások vették át. Ennek ok, hogy a digitális iránytó rendszerek jellemzően sokkal nagyobb flexibilitást kínálnak, hiszen az adott rendszer később is tetszőlegesen újraporgramozható. A beágyazott mikorvezérlők terjedésével pedig a költségük is rendkívül alacsony, illetve nem szabad megfeletkezni arról sem, hogy az ilyen, digitális irányító egységgel ellátott rendszerek komplexitása jellemzően sokkal alacsonyabb, hiszen a több 10-100 alaktrészből álló analóg szabályzót egy darab CPU helyettesíti, amely elvégiz a szükséges számításokat. A digitális és az analóg világ közti átmenet azonban kihívsáokat rejt magában. Míg az analóg jelek információtartalmát gyakorlatilag a zajterheltség és a sávszélesség határozza meg, addig digitális esetben számolnunk kell az ábrázolás \emph{bitmélységével}, ileltve a \emph{mintavételi idővel}. A digitális jelekre való konverzió nem veszteségmentes, az eredeti jel teljes valójába nem állíthaó vissza.

\subsection{Digitális jelek}

\subsection{Digitális számábrázolás}
\subsubsection{Számrendszerek}
\subsubsection{Negatív számok ábárzolása}
\subsubsection{Tört számok ábárázolása}
\subsubsection{Egyéb numerikus kódolások}
\paragraph{BCD kódolás}
\paragraph{Gray-kód}
\subsection{Digitális logika}
\subsection{Kombinációs logikai hálózatok}

\begin{example}[\quad \large 2 bites összeadó]

Készítsen 2 bites összeadót a megismert logikai kapuk segítségével! Az átvitelt nem kell figyelembe vennie!

\tcbline
\vspace{1mm}

\solution

\end{example}

\subsubsection{Alapvető logikai kapuk}
\subsubsection{Logikai függvények leírása}
\subsubsection{Logikai függvények egyszerűsítése}
\subsection{Sorrendi logikai hálózatok}

\begin{example}[\quad \large 2 bites összeadó]

Valósítsa meg az alábbi 1kimenetű 2 bemenetű funkciót NAND kapuk felhasználásával: Q=1, ha S=1, Q=0, ha S=0 és nR=0, egyébként Q megőrzi addigi értékét!

\tcbline
\vspace{1mm}

\solution

\end{example}





\vspace{-1.5mm}
\newpage